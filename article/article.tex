\documentclass[12pt]{article}

\usepackage{sbc-template}
\usepackage{graphicx,url}
\usepackage[utf8]{inputenc}
\usepackage[brazil]{babel}

\sloppy

\title{CreditAI: Sistema de Concessão de Crédito e Detecção de Risco
       com Técnicas de Inteligência Artificial}

\author{Diego Meinerz\inst{1}, Elias A. G. Miranda\inst{1},
        Fernanda T. Wammes\inst{1}, Matheus M. Burin\inst{1}}

\address{Universidade Regional do Noroeste do Estado do Rio Grande do Sul
  (UNIJUÍ)\\
  Santa Rosa -- RS -- Brazil, Departamento de Ciências da Computação
  \email{diego.meinerz@sou.unijui.edu.br, elias.miranda@sou.unijui.edu.br}\\
   \email{fernanda.wammes@sou.unijui.edu.br, matheus.burin@sou.unijui.edu.br}
}

\begin{document}

\maketitle

\begin{abstract}
This paper presents \textit{CreditAI}, a decision-support system for
credit granting and default risk detection in financial institutions.
The proposed model automates the analysis of loan applications by
combining four Artificial Intelligence techniques in a sequential
pipeline: (1) Depth-First Search (DFS) for persona classification using
decision trees, (2) Breadth-First Search (BFS) for credit limit
calculation exploring amount-installment state spaces, (3) Fuzzy Logic
with scikit-fuzzy for risk assessment, and (4) a Multi-Layer Perceptron
(MLP) neural network implemented in PyTorch for final approval decision.
The system receives customer profile data (income, credit score,
employment status, debt ratio, among others) and returns the approved
limit, risk level (low, medium or high) and the final status (approved,
rejected or pending review). Results indicate that the approach allows
more transparent, consistent and scalable credit analysis when compared
to purely manual processes.
\end{abstract}

\begin{resumo}
Este artigo apresenta o \textit{CreditAI}, um sistema de apoio à decisão
para concessão de crédito e detecção de risco de inadimplência em
instituições financeiras. O modelo proposto automatiza a análise de
propostas de empréstimo combinando quatro técnicas de Inteligência
Artificial em um pipeline sequencial: (1) Busca em Profundidade (DFS)
para classificação de personas usando árvores de decisão, (2) Busca em
Largura (BFS) para cálculo de limite explorando espaços de estados
valor-parcela, (3) Lógica Fuzzy com scikit-fuzzy para avaliação de
risco, e (4) uma Rede Neural Perceptron Multicamadas (MLP) implementada
em PyTorch para decisão final de aprovação. A solução recebe dados do
perfil do cliente (renda, score de crédito, status de emprego, taxa de
endividamento, entre outros) e retorna o limite aprovado, o nível de
risco (baixo, médio ou alto) e o status final (aprovado, reprovado ou
em análise). Os resultados indicam maior transparência, padronização e
escalabilidade em relação ao processo manual tradicional.
\end{resumo}

\section{Introdução}

A concessão de crédito é uma atividade central para bancos, cooperativas
e fintechs, mas envolve riscos relevantes de inadimplência e perda
financeira. Tradicionalmente, a análise de propostas era realizada por
analistas de crédito, de forma manual, a partir de documentos,
consultas a birôs e experiência prévia. Esse processo tende a ser lento,
subjetivo, custoso e suscetível a erros humanos, especialmente em
cenários de alto volume de solicitações.

Com a expansão do crédito digital e o surgimento de novas fintechs, há
uma demanda crescente por sistemas automatizados que apoiem a tomada de
decisão, reduzindo risco e aumentando a eficiência. Nesse contexto,
propomos o \textit{CreditAI}, um sistema que aplica quatro técnicas
distintas de Inteligência Artificial organizadas em um pipeline
sequencial para analisar pedidos de crédito, estimar limites e
classificar o risco de inadimplência de forma estruturada.

\section{Problema e Relevância}

O problema tratado neste trabalho pode ser resumido da seguinte forma:
dado o perfil de um cliente e os parâmetros de produtos de crédito
(empréstimo pessoal, cartão de crédito, garantia de veículo ou
imóvel), decidir se a proposta deve ser aprovada, reprovada ou
encaminhada para análise manual, bem como calcular o limite sugerido e
o risco associado.

Os principais desafios são:
(i) grande volume de solicitações;
(ii) dados por vezes incompletos ou inconsistentes;
(iii) necessidade de respostas rápidas;
(iv) cumprimento de políticas internas e exigências regulatórias;
(v) balanceamento entre inclusão financeira e gestão de risco.

A relevância do problema está na redução de perdas financeiras,
aumento da competitividade, maior inclusão financeira e melhoria da
rastreabilidade e explicabilidade das decisões de crédito.

\section{Dados Utilizados}

O \textit{CreditAI} utiliza dados estruturados do perfil do cliente,
organizados nos seguintes grupos:

\begin{itemize}
  \item \textbf{Dados demográficos:} idade, gênero e estado civil.
  
  \item \textbf{Dados financeiros:} renda mensal (\texttt{income}),
        score de crédito em escala de 0 a 1000
        (\texttt{credit\_score}), e taxa de endividamento como proporção
        da renda (\texttt{debt\_to\_income\_ratio}).
  
  \item \textbf{Dados de emprego:} status de emprego
        (\texttt{employment\_status}: empregado, autônomo, aposentado
        ou desempregado) e tempo no emprego atual em meses
        (\texttt{time\_at\_job\_months}).
  
  \item \textbf{Histórico bancário e de crédito:} existência de conta
        bancária (\texttt{has\_bank\_account}), restrições no BACEN
        (\texttt{has\_bacen\_restriction}), número de consultas de
        crédito recentes (\texttt{num\_credit\_inquiries}) e número de
        empréstimos ativos (\texttt{num\_existing\_loans}).
  
  \item \textbf{Dados da solicitação:} valor solicitado
        (\texttt{requested\_amount}), número de parcelas desejado
        (\texttt{requested\_installments}) e tipo de produto
        (\texttt{product\_type}).
\end{itemize}

Cada produto possui parâmetros específicos de valor mínimo/máximo,
número máximo de parcelas e taxa de juros base. 


\section{Modelo Proposto}

O \textit{CreditAI} organiza o processo de análise em quatro etapas
sequenciais, cada uma utilizando uma técnica de Inteligência Artificial
específica:

\subsection{Etapa 1: Filtro por Persona com Busca em Profundidade (DFS)}

A primeira etapa classifica o cliente em uma de três personas de crédito
-- \textit{Premium}, \textit{Standard} ou \textit{Basic} -- utilizando
uma árvore de decisão percorrida por \textbf{Busca em Profundidade
(Depth-First Search -- DFS)}.

A árvore é construída com nós de decisão baseados em condições sobre os
atributos do cliente. Cada nó interno representa uma condição
(por exemplo, ``Renda $\geq$ R\$ 10.000?''), com ramificações para
respostas verdadeiras e falsas. Os nós folha representam as personas
finais ou rejeição quando o cliente não se enquadra em nenhuma categoria.

A navegação na árvore inicia pela raiz e segue recursivamente pelos
ramos até alcançar uma folha:

\begin{enumerate}
  \item \textbf{Persona Premium:} Renda $\geq$ R\$ 10.000, emprego
        qualificado (empregado ou autônomo) e score $\geq$ 750.
        Limite máximo: R\$ 100.000, multiplicador de renda: 5x.
  
  \item \textbf{Persona Standard:} Renda $\geq$ R\$ 2.000, emprego
        qualificado e score $\geq$ 550. Limite máximo: R\$ 50.000,
        multiplicador de renda: 3x.
  
  \item \textbf{Persona Basic:} Renda $\geq$ R\$ 0.00, emprego
        qualificado ou aposentado, e score $\geq$ 0. Limite máximo:
        R\$ 20.000, multiplicador de renda: 2x.
\end{enumerate}

Se o cliente não atende aos critérios mínimos de nenhuma persona, a
proposta é rejeitada automaticamente nesta etapa.

\subsection{Etapa 2: Cálculo do Limite com Busca em Largura (BFS)}

O objetivo desta etapa é determinar o \textit{limite de crédito aprovado}
e o número de parcelas viável utilizando \textbf{Busca em Largura
(Breadth-First Search -- BFS)} para explorar o espaço de estados
(valor, parcelas).

O algoritmo funciona da seguinte forma:

\begin{enumerate}
  \item Calcula o limite base a partir da renda e do multiplicador da
        persona: \texttt{income\_limit = income * multiplier}.
  
  \item Aplica fatores de ajuste baseados em:
        \begin{itemize}
          \item Score de crédito (\texttt{score\_factor}): varia de
                0.7 a 1.2 dependendo da faixa de score.
          \item Status de emprego (\texttt{employment\_factor}):
                empregado (1.0), autônomo (0.95), aposentado (0.85),
                desempregado (0.5).
          \item Histórico de crédito (\texttt{history\_factor}):
                bonificação por empréstimos anteriores e penalização por
                alto endividamento.
        \end{itemize}
  
  \item Define o teto de busca (\texttt{search\_cap}) como o mínimo
        entre o limite da persona, limite do produto e limite ajustado
        pelos fatores.
  
  \item Inicia a BFS com estado inicial
        (\texttt{min\_amount}, \texttt{start\_installments}) e explora
        vizinhos:
        \begin{itemize}
          \item Incrementa o valor em passos fixos (ex: R\$ 500).
          \item Aumenta o número de parcelas para reduzir valor mensal.
        \end{itemize}
  
  \item Para cada estado, calcula a parcela mensal usando a fórmula
        Price:
        $$\text{PMT} = P \cdot \frac{r \cdot (1+r)^n}{(1+r)^n - 1}$$
        onde $P$ é o principal, $r$ a taxa mensal e $n$ o número de
        parcelas.
  
  \item Valida se a parcela respeita a regra de comprometimento máximo
        de renda (30\%). Se válido, atualiza o melhor valor encontrado.
  
  \item Continua a busca até esgotar os estados viáveis ou atingir o
        teto de busca.
\end{enumerate}

A BFS garante que o sistema explore sistematicamente as combinações de
valor e prazo, encontrando o maior limite aprovável que respeita as
restrições de capacidade de pagamento.

\subsection{Etapa 3: Avaliação de Risco com Lógica Fuzzy}

A terceira etapa estima o risco de inadimplência usando \textbf{Lógica
Fuzzy} implementada com a biblioteca \texttt{scikit-fuzzy}. Esta
abordagem permite lidar com a incerteza e subjetividade inerentes à
avaliação de risco.

O sistema fuzzy é composto por:

\begin{itemize}
  \item \textbf{Variáveis de entrada} (antecedentes):
        \begin{itemize}
          \item \texttt{credit\_score}: score de crédito (0--1000),
                com conjuntos fuzzy \textit{low}, \textit{med} e
                \textit{high}.
          \item \texttt{income}: renda mensal (0--50.000), com
                conjuntos \textit{low}, \textit{med} e \textit{high}.
          \item \texttt{debt\_ratio}: taxa de endividamento (0--1),
                com conjuntos \textit{low}, \textit{med} e \textit{high}.
          \item \texttt{employment\_time}: tempo de emprego em meses
                (0--120), com conjuntos \textit{short}, \textit{med} e
                \textit{long}.
          \item \texttt{inquiries}: consultas de crédito (0--20), com
                conjuntos \textit{few} e \textit{many}.
          \item \texttt{limit\_ratio}: razão entre valor solicitado e
                limite aprovado (0--1), com conjuntos \textit{low},
                \textit{med} e \textit{high}.
        \end{itemize}
  
  \item \textbf{Variável de saída} (consequente):
        \texttt{risk}: score de risco (0--1), com conjuntos
        \textit{low}, \textit{med} e \textit{high}.
  
  \item \textbf{Funções de pertinência}: funções trapezoidais e
        triangulares modelam a transição gradual entre conceitos
        (ex: score ``baixo'' vs ``médio''). O sistema fuzzy utiliza funções trapezoidais nos extremos das variáveis (garantindo região de pertinência máxima seguida de transição gradual) e funções triangulares para valores intermediários (com pico único no protótipo ideal). Essa combinação permite que um valor possua graus de pertinência em múltiplos conjuntos simultaneamente, ativando várias regras em paralelo e gerando decisões ponderadas que refletem a incerteza natural dos critérios de crédito.
  
  \item \textbf{Regras fuzzy}: 15 regras do tipo SE-ENTÃO capturam o
        conhecimento especialista, por exemplo:
        \begin{itemize}
          \item SE score é \textit{high} E dívida é \textit{low} ENTÃO
                risco é \textit{low}.
          \item SE score é \textit{low} E dívida é \textit{med} ENTÃO
                risco é \textit{high}.
          \item SE dívida é \textit{high} E renda é \textit{low} ENTÃO
                risco é \textit{high}.
        \end{itemize}
\end{itemize}

O processo de inferência fuzzy funciona como segue:

\begin{enumerate}
  \item \textbf{Fuzzificação:} os valores numéricos das entradas são
        convertidos em graus de pertinência nos conjuntos fuzzy.
  \item \textbf{Inferência:} as regras são avaliadas usando operadores
        fuzzy (AND = mínimo, OR = máximo).
  \item \textbf{Agregação:} os resultados das regras ativadas são
        combinados.
  \item \textbf{Defuzzificação:} o conjunto fuzzy de saída é
        convertido em um valor numérico.
\end{enumerate}

O score de risco resultante (0--1) é mapeado em três níveis:

\begin{itemize}
  \item \textbf{Baixo risco:} risco $< 0.40$
  \item \textbf{Médio risco:} $0.40 \leq$ risco $< 0.70$
  \item \textbf{Alto risco:} risco $\geq 0.70$
\end{itemize}

O sistema também gera gráficos das curvas de pertinência com o score
calculado marcado, facilitando a interpretação e auditoria das decisões.

\subsection{Etapa 4: Decisão Final com Rede Neural (PyTorch)}

A etapa final utiliza uma \textbf{Rede Neural Perceptron Multicamadas
(MLP)} implementada em \textbf{PyTorch} para determinar a decisão final:
\textit{approved}, \textit{pending\_review} ou \textit{rejected}.

A arquitetura da rede é:

\begin{itemize}
  \item \textbf{Camada de entrada:} 10 neurônios correspondentes aos
        atributos normalizados:
        \begin{itemize}
          \item Idade normalizada: $(idade - 18) / (75 - 18)$
          \item Score normalizado: $score / 1000$
          \item Renda normalizada: $\min(1, \ln(1 + renda) / \ln(50001))$
          \item Taxa de endividamento (0--1)
          \item Emprego (binário: 1 se empregado/autônomo, 0 caso
                contrário)
          \item Conta bancária (binário)
          \item Consultas normalizadas: $\min(1, consultas / 10)$
          \item Empréstimos normalizados: $\min(1, empréstimos / 5)$
          \item Score de risco fuzzy (0--1)
          \item Razão limite: $\min(1, valor\_solicitado / limite)$
        \end{itemize}
  
  \item \textbf{Camada oculta:} 16 neurônios com função de ativação ReLU.
  
  \item \textbf{Camada de saída:} 3 neurônios (logits) correspondentes
        às classes \textit{approved}, \textit{pending} e
        \textit{rejected}.
\end{itemize}
Utilizou-se duas funções de ativação:

ReLU na camada oculta: evita saturação de gradientes, treina rápido em dados tabulares e lida bem com valores não normalizados simétricos.

Softmax na saída: transforma os logits em probabilidades somando 1 para as três classes.

Os pesos da rede são inicializados com heurísticas de negócio alinhadas
às regras de rotulagem dos dados sintéticos, garantindo convergência
estável durante o treinamento. A decisão final é obtida aplicando
\texttt{softmax} aos logits e selecionando a classe com maior
probabilidade.

\subsubsection{Treinamento da Rede}

O sistema implementa um pipeline completo de treinamento:

\begin{enumerate}
  \item \textbf{Geração de dados sintéticos:} método
        \texttt{generate\_dataset\_jsonl} cria exemplos de treino com
        1000 amostras por padrão, usando regras de negócio para
        rotulação:
        \begin{itemize}
          \item \textit{Rejected} se: risco fuzzy $> 0.75$ OU score $< 500$
                OU endividamento $> 0.55$.
          \item \textit{Pending} se: risco $\geq 0.45$ OU razão limite
                $> 0.95$ OU desempregado.
          \item \textit{Approved} caso contrário.
        \end{itemize}
  
  \item \textbf{Treinamento:} método \texttt{train\_from\_jsonl} realiza
        o treinamento supervisionado usando:
        \begin{itemize}
          \item Otimizador: Adam com learning rate de $10^{-3}$ e
                weight decay de $10^{-4}$.
          \item Função de perda: Cross-Entropy Loss.
          \item Batch size: 64.
          \item Épocas: 30 (padrão).
        \end{itemize}
  
  \item \textbf{Rastreamento com MLflow:} o sistema integra-se com
        MLflow para registrar:
        \begin{itemize}
          \item Hiperparâmetros (learning rate, epochs, batch size).
          \item Métricas de perda por época.
          \item Artefatos (pesos do modelo, dados de treino).
        \end{itemize}
  
  \item \textbf{Salvamento e carregamento:} os pesos treinados são
        salvos em \texttt{models/approval\_mlp.pt} e carregados
        automaticamente na próxima inicialização.
\end{enumerate}

A rede neural permite capturar interações não-lineares complexas entre
os atributos que regras manuais dificilmente modelariam, além de
fornecer probabilidades calibradas para cada decisão.

\section{Pipeline de Execução}

O \textit{CreditAI} executa as quatro etapas sequencialmente, com saída
antecipada em caso de rejeição:

\begin{enumerate}
  \item Se a \textbf{Etapa 1} (DFS) não encontrar persona válida, a
        proposta é rejeitada com motivo \texttt{PERSONA\_FILTER}.
  
  \item A \textbf{Etapa 2} (BFS) calcula o limite aprovado e fatores
        relevantes.
  
  \item A \textbf{Etapa 3} (Fuzzy) avalia o risco de inadimplência.
  
  \item A \textbf{Etapa 4} (RNA) toma a decisão final considerando todos
        os dados anteriores:
        \begin{itemize}
          \item Se \textit{approved}: retorna valor aprovado, parcelas e
                valor mensal.
          \item Se \textit{rejected}: define o motivo como
                \texttt{HIGH\_RISK} se score fuzzy $> 0.65$, caso
                contrário \texttt{OTHER}.
          \item Se \textit{pending}: encaminha para análise manual.
        \end{itemize}
\end{enumerate}

O sistema retorna um objeto \texttt{CreditAnalysisResult} completo com
todos os detalhes das quatro etapas, permitindo auditoria e
explicabilidade da decisão.

\section{Resultados e Discussão}

A implementação do \textit{CreditAI} demonstra a viabilidade de integrar
múltiplas técnicas de IA em um pipeline de análise de crédito. Cada
técnica contribui com capacidades específicas:

\begin{itemize}
  \item \textbf{DFS (Etapa 1):} Garante classificação eficiente em
        árvore de decisão com tempo $O(h)$, onde $h$ é a altura da
        árvore. Permite segmentação clara de clientes e aplicação de
        políticas diferenciadas por persona.
  
  \item \textbf{BFS (Etapa 2):} Explora sistematicamente o espaço de
        soluções valor-parcela, garantindo que a melhor combinação
        viável seja encontrada. Tempo $O(V)$ onde $V$ é o número de
        estados visitados (limitado pelo teto de busca e passo).
  
  \item \textbf{Lógica Fuzzy (Etapa 3):} Trata incerteza e conceitos
        vagos (``score razoável'', ``dívida alta'') de forma natural,
        produzindo scores contínuos mais informativos que classificações
        binárias. Facilita explicação das decisões através das regras
        ativas.
  
  \item \textbf{RNA/PyTorch (Etapa 4):} Captura padrões complexos e
        não-lineares, aprendendo com dados sintéticos que refletem
        políticas de negócio. O treinamento com backpropagation permite
        ajuste fino dos pesos. Fornece probabilidades calibradas para
        cada classe de decisão.
\end{itemize}

\subsection{Vantagens da Abordagem}

\begin{itemize}
  \item \textbf{Modularidade:} cada etapa pode ser ajustada ou
        substituída independentemente.
  \item \textbf{Explicabilidade:} as três primeiras etapas são baseadas
        em regras interpretáveis; a RNA fornece probabilidades que
        auxiliam na explicação.
  \item \textbf{Escalabilidade:} o sistema processa requisições em
        tempo constante após carregamento dos modelos, adequado para
        alto volume.
  \item \textbf{Rastreabilidade:} integração com MLflow permite
        versionamento e auditoria completa dos modelos e decisões.
\end{itemize}

\section{Considerações Finais}

O \textit{CreditAI} demonstra como técnicas clássicas de IA (busca em
grafos, lógica fuzzy) podem ser integradas com aprendizado de máquina
moderno (redes neurais com PyTorch) para criar um sistema de análise de
crédito robusto e explicável. A organização em pipeline sequencial com
quatro etapas distintas permite balancear eficiência computacional,
interpretabilidade e capacidade preditiva.

O projeto serve como prova de conceito para instituições financeiras
interessadas em automatizar e padronizar seus processos de concessão de
crédito, reduzindo custos operacionais e tempo de resposta enquanto
mantém controle sobre políticas de risco. A arquitetura modular facilita
adaptações para diferentes produtos, segmentos de mercado e requisitos
regulatórios.


\bibliographystyle{sbc}
\bibliography{sbc-template}
\cite{cerf2024balancedbfs}
\cite{chang2022mlp}
\cite{jurkovic2004stochasticfuzzy}
\cite{tarjan2022strongcomponents}


\end{document}
